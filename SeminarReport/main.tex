\documentclass[]{rptuseminar}

% Specify that the source file has UTF8 encoding
\usepackage[utf8]{inputenc}
% Set up the document font; font encoding (here T1) has to fit the used font.
\usepackage[T1]{fontenc}
\usepackage{lmodern}

% Load language spec
\usepackage[american]{babel}
% German article --> ngerman (n for »neue deutsche Rechtschreibung«)
% British English --> english

% Ffor bibliography and \cite
\usepackage{cite}

% AMS extensions for math typesetting
\usepackage[intlimits]{mathtools}
\usepackage{amssymb}
% ... there are many more ...


% Load \todo command for notes
\usepackage{todonotes}
% Sebastian's favorite command for large inline todonotes
% Caveat: does not work well with \listoftodos
\newcommand\todoin[2][]{\todo[inline, caption={2do}, #1]{
		\begin{minipage}{\linewidth-1em}\noindent\relax#2\end{minipage}}}

% Load \includegraphics command for including pictures (pdf or png highly recommended)
\usepackage{graphicx}

% Typeset source/pseudo code
\usepackage{listings}

% Load TikZ library for creating graphics
% Using the PGF/TikZ manual and/or tex.stackexchange.com is highly adviced.
\usepackage{tikz}
% Load tikz libraries needed below (see the manual for a full list)
\usetikzlibrary{automata,positioning}

% Load \url command for easier hyperlinks without special link text
\usepackage{url}

% Load support for links in pdfs
\usepackage{hyperref}

% Defines default styling for code listings
\definecolor{gray_ulisses}{gray}{0.55}
\definecolor{green_ulises}{rgb}{0.2,0.75,0}
\lstset{%
  columns=flexible,
  keepspaces=true,
  tabsize=3,
  basicstyle={\fontfamily{tx}\ttfamily\small},
  stringstyle=\color{green_ulises},
  commentstyle=\color{gray_ulisses},
  identifierstyle=\slshape{},
  keywordstyle=\bfseries,
  numberstyle=\small\color{gray_ulisses},
  numberblanklines=false,
  inputencoding={utf8},
  belowskip=-1mm,
  escapeinside={//*}{\^^M} % Allow to set labels and the like in comments
}

% Defines a custom environment for indented shell commands
\newenvironment{displayshellcommand}{%
	\begin{quote}%
	\ttfamily%
}{%
	\end{quote}%
}

%%%%%%%%%%%%%%%%%%%%%%%%%%%%%%%%%%%%%%%%%%%%%%%%%%%%%%%%%%%%%%%%%%%%%%%%%%%%%%%

\title{Liquid Haskell}
\event{Seminar: Programming Languages in Winter term 2024/2025}
\author{Mehran Shahidi, Saba Safarnezhad
  \institute{Rheinland-Pfälzische Technische Universität Kaiserslautern-Landau, Department of Computer Science}}

%%%%%%%%%%%%%%%%%%%%%%%%%%%%%%%%%%%%%%%%%%%%%%%%%%%%%%%%%%%%%%%%%%%%%%%%%%%%%%%
\begin{document}
%%%%%%%%%%%%%%%%%%%%%%%%%%%%%%%%%%%%%%%%%%%%%%%%%%%%%%%%%%%%%%%%%%%%%%%%%%%%%%%

\maketitle

%%%%%%%%%%%%%%%%%%%%%%%%%%%%%%%%%%%%%%%%%%%%%%%%%%%%%%%%%%%%%%%%%%%%%%%%%%%%%%%

\begin{abstract}
  This report gives a brief overview of Liquid Haskell, a tool that extends Haskell with refinement types. Refinement types are types that extends expressiveness of Haskell types systems by providing predicates that can verify
  invarients of the program. This report explains briefly how SMT solvers leveraged by Liquid Haskell and how to use Liquid Haskell by providing some examples. Finally, we discuss the limitations of Liquid Haskell and compare it with other tools.
\end{abstract}

%%%%%%%%%%%%%%%%%%%%%%%%%%%%%%%%%%%%%%%%%%%%%%%%%%%%%%%%%%%%%%%%%%%%%%%%%%%%%%

\section{Introduction}
\label{sec:introduction}

% E.~g.~ 
% ``quoting'' is done by using two backticks and two single quotes

\section{Background}
\label{sec:background}

\section{Working with LiquidHaskell}
  
  
\section{Example Application}

\section{Conclusions, Results, Discussion}
\label{sec:conclusions}

% \begin{figure}
% \begin{lstlisting}[language=Scala,numbers=left]
% def sort[T](list: List[T])(implicit ord: Ordering[T]): List[T] = {
%   list match {
%     case Nil => Nil
%     case x :: xs =>
%       // partition list based on pivot element x
%       val (lo, hi) = xs.partition(ord.lt(_, x))  //*\label{srcline:algcmp}
%       sort(lo) ++ (x :: sort(hi))
%   }
% }
% \end{lstlisting}
% \caption{An example algorithm. Do you recognize it? Line~\ref{srcline:algcmp} is crucial!}
% \label{alg:example}
% \end{figure}

\section{Bibliography}

%%%%%%%%%%%%%%%%%%%%%%%%%%%%%%%%%%%%%%%%%%%%%%%%%%%%%%%%%%%%%%%%%%%%%%%%%%%%%%%
\newpage
\nocite{*}
\bibliographystyle{eptcs}
\bibliography{references}

%%%%%%%%%%%%%%%%%%%%%%%%%%%%%%%%%%%%%%%%%%%%%%%%%%%%%%%%%%%%%%%%%%%%%%%%%%%%%%%
\end{document}
%%%%%%%%%%%%%%%%%%%%%%%%%%%%%%%%%%%%%%%%%%%%%%%%%%%%%%%%%%%%%%%%%%%%%%%%%%%%%%%
